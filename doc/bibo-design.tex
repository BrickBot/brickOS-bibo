\documentclass[twocolumn]{article}
\begin{document}
  \title{Introduction to the bibo kernel}
  \author{Jochen Hoenicke}
  \maketitle

  \tableofcontents

  \section{Task Management}

  \subsection{Priorities}
  
  The bibo kernel contains a priority based scheduler.  The following
  table shows the priorities ordered by highest priority first.
  
  \begin{center}
    \begin{tabular}{|c|}
      \hline
      Interrupts \\
      Handlers \\
      \hline
      \tt PRIO\_HIGHEST \\
      $\vdots$ \\
      \tt PRIO\_LOWEST \\
      \hline
      Idle thread\\
      \hline
    \end{tabular}
  \end{center}

  Tasks with higher priority always preempt tasks with lower priority.
  Tasks with the same priority have to cooperate by calling yield().  Tasks
  with lower priority are only executed when all tasks with a higher priority
  are waiting.
  
  Interrupts can preempt all code, even code that locks the kernel.  Therefore
  they must not call functions like \verb|malloc|/\verb|free|, \verb|wakeup|,
  etc.  The only safe function is \verb|enqueue_handler_irq|.  This function
  enqueues a handler that will run as soon as the interrupt is finished and
  the kernel lock is released.  Interrupts can run in any task context; hence
  they should not use much stack.
  
  Handlers are short functions that run with the second highest priority.
  They cannot preempt each other; they run sequentially.  Handlers do
  not preempt code that locks the kernel.  Instead they are delayed until the
  kernel lock is released.  The effect is that locked code runs on the same
  priority as handlers.  Handlers always run in the idle task and have their
  own stack space.

  You can register handlers for various purposes.  For example
  \verb|lnp_addressing_set_handlers()| can be used to register
  a handler that is called whenever a lnp packet is received.  

  \subsection{Kernel Lock}

  For atomic operations you should grab the global kernel lock, with
  \verb|grab_kernel_lock()| and release it afterwards with
  \verb|release_kernel_lock()|.  While you grab the lock, your process is
  given the handler priority.  Hence, no task switch can occur and no handler
  can be called.  The kernel lock can be nested: it is safe to call a function
  the grabs the kernel lock even if you have already grabbed it yourself.

  Because interrupts preempt code that locked the kernel you cannot grab the
  lock from an interrupt.  However, it is perfectly fine for handlers; it is
  not necessary though as handlers always hold the lock.  You can do almost
  anything in handlers except calling \verb|wait| or a function that uses
  \verb|wait| like the \verb|lnp_XXX_write| routines.

  The following functions must always be called with grabbed kernel lock.

  \begin{center}
  \begin{tabular}{|c|}
    \hline
    \verb|add_timer|, \verb|remove_timer|,  \\
    \verb|add_to_waitqueue|, \verb|remove_waitqueue|, \\
    \verb|wait|, \verb|wait_timeout|, \\
    \verb|wakeup_single|, \verb|wakeup|\\
    \hline
  \end{tabular}
  \end{center}
  
  Many functions use kernel locks internally:
  \begin{quote}
    \verb|malloc|, \verb|free|,
    \verb|sem_wait|, \verb|sem_trywait|, \verb|sem_post|
    \verb|execi|, \verb|msleep|, \verb|yield|
  \end{quote}

  \subsection{Wait Queues}

  A process that waits for an event should do the following: First grab
  the kernel lock. This should prevent that the event occurs before it is
  ready waiting for it.  Then check if the event has not yet occured.  
  Register the process on the wait queues, call \verb|wait()| or
  \verb|wait_timeout()|, and remove process from all wait queues afterwards.
  Finally release kernel lock.  Here is the pseudo code:

  \begin{verbatim}
waitqueue_t entry[2];
grab_kernel_lock();
if (event_has_not_occured()) {
   add_to_waitqueue(&queue1, &entry[0]);
   add_to_waitqueue(&queue2, &entry[1]);

   wait();

   remove_from_waitqueue(&entry[0]);
   remove_from_waitqueue(&entry[1]);
}
release_kernel_lock();
\end{verbatim}

  To wake up tasks on a wait queue you can just run \verb|wakeup| or
  \verb|wakeup_single|.  This function must be called with grabbed kernel
  lock.  Normally, it is invoked from a handler.

  \subsection{Timers}
  
  Timers invoke handler functions after a few milli seconds.  This handler
  function may also reschedule its timer to be called periodically.  Like all
  handlers they run with handler priority on the idle. The following timer
  functions methods must be called with grabbed kernel lock.

  The function \verb|add_timer(delay, timer)| adds a timer to be scheduled
  after \verb|delay| milli seconds.  The members \verb|code| and \verb|data|
  must be set appropriately.  All other members are filled by
  \verb|add_timer|.
  
  The function \verb|remove_timer(timer)| removes a timer.  It is guaranteed
  that the timer will not be scheduled after this call.

  The function \verb|get_timer_count(timer)| can be used to check whether a
  timer is running and how long it will take to be scheduled.

  When a timer handler is called the timer is already dequeued from the timer
  list.  The timer can be added by the handler, again, to implement periodic
  behaviour.


  \subsection{Internals}

  Whenever an interrupt is about to finish, it checks if an handler was
  enqueued or we need to reschedule a different process.  This is remembered
  in the high bit of \verb|kernel_lock|.  In that case, the idle thread is
  activated by \verb|tm_switcher(&td_idle)|.  The idle thread runs all
  handlers that are enqueued.  Then it activates the first thread on the run
  queue, which is always the thread with highest priority.  If there is no
  thread on the run queue the idle thread issues the \verb|sleep| instruction.

  The \verb|kernel_lock| is a single 8 bit variable.  The highest bit is
  normally set.  It gets cleared when handlers are enqueued or a new task is
  to be scheduled.  The variable is increased by \verb|grab_kernel_lock()|.
  This will tell the interrupts that the kernel is locked and no task switch
  to idle thread should occur.  The macro \verb|release_kernel_lock()|
  decreases the variable and checks for zero.  If it is zero, the lock was
  released completely and there are handlers enqueued or a task switch
  requested.  Therefore the macro switches to the idle thread by invoking
  \verb|tm_switch_handlers()|.  The idle thread will take care of running
  handlers and scheduling a new task.

  The active timers are kept on a linked list sorted by activation time.  The
  activation delay of the first timer is held in a global variable.  The
  global time interrupt \verb|secondary_handler|, which is triggered by
  \verb|OCIB| decrements this variable.  When the variable gets zero, the
  \verb|run_timers| handler is enqueued.  This handler calls all timer
  handlers that have expired and removes them from the list.

  \section{Semaphores}
  The bibo kernel provides the classical semaphore operations invented by
  Dijkstra.  The names of the operations mostly follow the POSIX 1003.1b
  standard.  The semaphore is a shared counter that can be used for
  synchronisation of parallel processes.  The \verb|sem_wait| function waits
  until the counter gets non-zero and atomically decreases it.  The
  \verb|sem_post| operation atomically increases the counter and, if it was
  zero, wakes up the process with highest priority waiting for the semaphore.
  Furthermore there is the non-blocking \verb|sem_trywait| methods that works
  just like \verb|sem_wait| but return an error if the semaphore is zero
  instead of blocking.  The \verb|sem_getvalue| operation returns the current
  value of the semaphore counter.

  \section{Memory Manager}
  The bibo kernel supports dynamic memory via standard \verb|malloc()| and
  \verb|free()| calls.  It uses a very simple \emph{first fit} algorithm and
  provides no mechanism to avoid memory fragmentation.  When a block is freed
  it merges it with adjacent free blocks.  All memory blocks are word aligned
  and rounded up to an even size.  The overhead of the memory manager
  structures is two words (four bytes) per memory block.

  The memory blocks are all on a large linked list.  A memory block starts
  with the following header:
  \begin{center}
  \begin{tabular}{r|ll|}
    \cline{2-3}
    0: & previous block & (only if previous free)\\
    \cline{2-3}
    2: & next block  &(bit 0: prev free)   \\
    4: & owner &(0 for free block) \\
    \cline{2-3}
    6: & next free block & (only if this free)\\
    \cline{2-3}
  \end{tabular}
  \end{center}
  The previous block pointer is only present if the previous block is free.
  It points to the header of the previous block. This is used by \verb|free()|
  to quickly merge adjacent free blocks.  If the previous block contains data,
  this data will use this field and it must not be touched.

  The next block pointer contains a pointer to the next memory block, which
  can be free, used or reserved.  Thus the memory blocks form a linked
  list. The list is always kept sorted by block address. The lowest bit of the
  address is always zero, so we can reuse this bit for other purposes.  We
  store whether the previous block is free.

  The owner field contains the thread id of the owner.  It is \verb$0$ for
  free blocks and \verb$0xffff$ for reserved blocks.  This field is used to
  clean up orphaned memory when a thread is killed.

  The next free block field points to the next free block, but only if the
  current block is free.  Thus the free blocks form another singly linked list
  that is a sub list of the list of all memory blocks.  The list is always
  kept sorted by block address.  If the block is used this field contains the
  first word of the payload data.

  For used blocks the memory header has only two words storage overhead.  The
  other fields contain payload data of this or the previous block.  This makes
  the minimum size of the payload 2 words.  If a smaller block was requested
  it will be rounded up to this size.
\end{document}
